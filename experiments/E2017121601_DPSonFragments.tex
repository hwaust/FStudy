

\section{E2017120501: Data Partitioning on Fragmented XML Data}

This report is about the experiments to evaluate the performance 
of applying data partitioning strategy on fragmented XML data. 


\subsection{A Brief Introduction}



\subsection{Settings}

The settings of this experiments are listed below.

\paragraph{Hardware}

There were 1 master (HaoDesk) and 16 workers (matsu-lab00 -- matsu-lab15) 
used in the experiments listed in \ref{app:xqueries}.



\paragraph{Software}
The input XML data set was xmark600(see \ref{app:xqueries}), which 
was fragmented into 16 fragments with maxisze=4M.
On each worker, there ran a BaseX server (Verion: 8.6.7). 
For each fragment, a BaseX database instance, namely `xmark600\_16\_4M\'
was created in main memory on each worker.
For the data partition, the number of partions(P) were  1, 2, 3, 4, 8.
The intermediate database for holding the results of prefix
query were named  `xmark600\_16\_4M\_tmp'.
 

\paragraph{XQuery Expressions}

Two XPath expressions XQ1 and XQ2 for prefix and suffix queries 
are listed in Section~\ref{app:xqueries}.



\subsection{Confirming Correctness}

The number of hit nodes, the order of hit nodes and results size of
the final output of queries have been checked and confirmed to be correct. 
All the successfully evaluated queries 
have the same number of hit nodes in original order. Some may be 
different in size, such as xm3a.dps. Its original result size
was 922,270,281, but in this experiment, it is 883,253,777.
This dramatical difference is caused by line-break. The previous 
experiments were done on Windows, while the current on Linux. 
Since BaseX using `$\backslash$r$\backslash$n' on Windows, while
 `$\backslash$n' on Linux for line-break, there is one byte 
difference for each new line. We also found that the query has 
6,502,751 hit nodes and  there are six lines in every hit node, 
such as:\\
\verb|<bidder>|\\
\verb|  <date>08/04/1999</date>|\\

\verb|  <time>11:15:36</time>|\\
\verb|  <personref person="person17793"/>|\\
\verb|  <increase>7.50</increase>|\\
\verb|</bidder>|\\

We then have 883,253,777 + 6,502,751 * 6 - 922,270,281 = 2. These two bytes
are extra '$\backslash$n'. So the sizes are the same.



\subsection{Discussion on Queries}


\paragraph{xm1.dps}

Since the results of xm1.dps are larger than the memory of 
HaoDesk. This query was not evaluated. This is a design 
choice about how to process the results. Based the previous 
evaluation, one possible way is to stored the unordered intermediate 
results in files and then concatenate these files.

\paragraph{xm2.dps}

Not tested for two reasons. Firstly, the prefix query of xm2.dps (as well
as xm4.dps) cannot be processed by the current XQuery expression XQ1
(will be modifed and tested soon). Secondly, there is no results as 
shown in experiment E20170401. One proposal is to  make a minimal 
change to the query, such as change \texttt{category52} to 
 \texttt{category324329}, which exists in XMark600.




\paragraph{xm3.dps}

The results of xm3.dps are shown in Table~\ref{tab:xm3dps}.
(xm3.org = 63.32s) 


\paragraph{xm4.dps}
Not tested for the same reason as the first one of xm2.dps.

\paragraph{xm5.dps}

The query xm5a.dps was the targe query in the previous design. 
However, due to the insufficient number of nodes in the results of prefix 
query to be allocated to 16 
computers\footnote{
Detailed explanation: An error message 
``database 'xmark600\_16\_4M\_tmp' has no node with pre value 5.'' was
encountered when executing a suffix query on an intermediate database.
The database had the content of  \texttt{<root><part> ...</part></root>} , 
where there is only one \texttt{part} node. 
In a suffix query when P = 2, it is then to process ``pre value 5'' in XQ2, 
which refers to the second \texttt{part} node. However, since there is 
no second \texttt{part} node, the error occurred.
}
, I changed it to xm5b.dps (xm5c.dps was also evaluated, but 
it took 1480s (P=4), which is too long and thus ignored). 
The results of xm5b.dps is listed in Table~\ref{tab:xm5b.dps}
(xm5.dps = 75.05s). 


\paragraph{xm6.dps}

Failed to evaluate them with data partitioning. The reason is 
the same as xm5a.dps (both \texttt{/site/regions} and 
\texttt{/site/regions/*[...]/item} were tested).


\begin{table}[]
	\centering
	\caption{Execution time with related information}
	\label{tab:exetime}
	\begin{tabular}{|c|c|c|c|c|c|c|c|c|c|c|}
		\hline
		\multirow{2}{*}{Key} & \multirow{2}{*}{\begin{tabular}[c]{@{}c@{}}Prefix\\ (s)\end{tabular}} & \multicolumn{5}{c|}{Suffix (s)} & \multirow{2}{*}{\begin{tabular}[c]{@{}c@{}}Merge \\ (s)\end{tabular}} & \multirow{2}{*}{\begin{tabular}[c]{@{}c@{}}Original\\ (s)\end{tabular}} & \multicolumn{2}{c|}{Results Size} \\ \cline{3-7} \cline{10-11} 
		&                                                                       & P=1 & P=2  & P=3  & P=4  & P=8  &                                                                       &                                                                         & hit nodes(M)       & in MiB       \\ \hline
		xm1.dps              &                                                                       &     &      &      &      &      &                                                                       & 3122                                                                    & 5.5                & 57267        \\ \hline
		xm2.dps              &                                                                       &     &      &      &      &      &                                                                       & 0.01                                                                    & 0                  & 0            \\ \hline
		xm3.dps              & 0.3                                                                   & 113 & 58.5 & 30.1 & 20.2 & 13.5 & 6                                                                     & 63.3                                                                    & 6.5                & 880          \\ \hline
		xm4.dps              &                                                                       &     &      &      &      &      &                                                                       & 101                                                                     & 0.6                & 1511         \\ \hline
		xm5.dps              & 0.3                                                                   & 243 & 67.5 & 48.9 & 37.8 & 31.8 & 10                                                                    & 74.9                                                                    & 35.9               & 944          \\ \hline
		xm6.dps              &                                                                       &     &      &      &      &      &                                                                       & 70                                                                      & 1.3                & 1289         \\ \hline
	\end{tabular}
\end{table}


 
%\subsection{Conclusion}
%
%In general there are three problems or design choices:
%
%\begin{itemize}
%\item xm1.dps returns larger results than memory.
%\item some queries cannot be successfully evaluated.
%\item some successfully evaluated queries are too slow in performance.
%\end{itemize}


\subsection{Efficiency of Parsing Intermediate Results of Suffix Query}
A important method in the implemenation that affects the whole performance is 
\texttt{basex.PreValueReceiver.process(InputStream input)} used 
to parse the received results of suffix query, (i.e. PRE value + 
content). It takes the results of suffix query and returns an 
QueryResultsPre instance with a list of PRE values and a list of 
string content (of the same size). A simple experiment were done to evaluate 
the parsing speed. In the experiment, it took 465 ms to parse 
52,757 KiB data with 704,430 nodes, i.e. it can process 100 MiB 
data per second, which basically reach the maximum network 
speed and thus should be sufficient for not being a bottleneck.




%\begin{table}
%	\centering
%	\label{tab:xm5b.dps}
%	\caption{Execution time for xm5b.dps}
%	\begin{tabular}{c|c|c|c|c|c|c|c}
%		\hline
%		query    &  P=1  &  P=2   &  P=3  &  P=4 &   P=6 &    P=8 & P=12   \\
%		\hline
%		prefix    &   \multicolumn{6}{|c}{$\approx$ 0.3s}    \\
%		\hline
%		suffix    & 243s & 67.5s &  48.9s & 37.8s & 33.4s&   31.8s &   28.2s \\ 
%		\hline
%		merge     &   \multicolumn{6}{|c}{$\approx$ 10s}    \\
%		\hline
%	\end{tabular} 
%\end{table} 
%
%\begin{table}
%	\centering
%	\label{tab:xm3dps}
%	\caption{Execution time for xm3.dps}
%	\begin{tabular}{c|c|c|c|c|c|c|c}
%		\hline
%		query    &  P=1  &  P=2   &  P=3  &  P=4 &   P=6 &    P=8 & P=12   \\
%		\hline
%		prefix    &   \multicolumn{6}{|c}{$\approx$ 0.3s}    \\
%		\hline
%		suffix    & 113.2s&  58.5s &  30.1s&  20.2s & 15.8s&   13.5s & 11.3s \\ 
%		\hline
%		merge     &   \multicolumn{6}{|c}{$\approx$ 6s}    \\
%		\hline
%	\end{tabular} 
%\end{table} 


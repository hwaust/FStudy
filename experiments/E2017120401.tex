\subsection{E2017120401: Baseline Performance Evaluation}
 
\subsubsection{Purpose}
To understand the baseline performance by using
the queries in ADBIS submission for XMark600.
 
\subsubsection{Settings} 

\begin{itemize}

\item \textbf{Hardware} HaoDesk (see \ref{HaoDesk})\\
\item \textbf{Software} BaseX 6.8.7, Java 1.8.0\_151(x64).\\
\item \textbf{XML Dataset} XMark600.xml (see Table~\ref{table:xmark}), 
from which a BaseX databases `xmark600' is created in a BaseX server using command:\\
\verb|create db xmark600 xmark600.xml|
\item \textbf{Queries} xm1.org -- xm6.org (see ).

\end{itemize}


\subsubsection{Experiment Design}  

A BaseX instance Server first runs in server mode started by the following
command on HaoDesk.

\verb| java -Xmx4g -xms2g -cp BaseX897.jar org.basex.BaseXServer|

Note the databases in Server are NOT in main memory mode.

Then, a java program APP runs on HaoDesk in charge of sending an input query to
Server via local network and saving results returned from Server to memory(short
for mem) or disk depending on settings. An input query Query that will be
processed in APP is first rewritten into the following XQuery expression:

\verb|for $node in db:open('xmark600')Query return $node|

The results are stored either in memory or on disk depending on APP's settings:
a) memory means the results are stored in memory and then discarded after the
experiments. b) disk means the results are stored in disk and will be preserved
after the experiments. One more thing, the maximum memory for APP was set to 12 GB.


\subsubsection{Experiment Results}

\textbf{Timing} The execution time is measured in APP. The time period between
starting sending a query and finishing receiving the results is measured as
execution time. Each query is evaluated 5 times. 

\textbf{Process Results}
We removed the results of the first run and take the average of the rest as
the final execution time listed in Table~\ref{table:E2017120401_1}.


\textbf{Original Data}

All the original results containing execution time of queries and scripts used 
in the experiments  are stored to \texttt{experiments/E2017120401} relative to
the current folder that stores this report.\\

Note: The result of xm2.org is always empty (still under investigation).


\begin{table}[t]
	\caption{Experiment Results of E2017120401.}
	\label{table:E2017120401_1}
	\centering
	\begin{tabular}{c|c|c|r}
 		\hline \hline
 query  & storage & time(s)  &   result size  \\
 \hline \hline
 xm1.org &  disk   & 3191.60  & 60,048,845,586 \\
         & memory  &    N/A   &  \\
 \hline
 xm2.org &  disk   &    0.01  &              0 \\
         & memory  &    0.01  &  \\
  \hline
 xm3.org &  disk   &  71.25  &    922,270,281 \\
         & memory  &  73.34  &  \\
  \hline
 xm4.org &  disk   &  113.05  &  1,583,959,305 \\
         & memory  &  113.84  &  \\
  \hline
 xm5.org &  disk   &  83.59  &    989,346,990 \\
         & memory  &  88.75   &   \\
  \hline
 xm6.org &  disk   &  78.42   &  1,351,708,787 \\
         & memory  &  78.90   &   \\
 \hline \hline
	\end{tabular}
\end{table}





\subsubsection{Observations}

\begin{itemize}
\item \textbf{Storage has small influence on execution time}
 
We noticed one thing that the execution time is pretty similar for
all the available queries. This is because the bottleneck is on the
worker's side but not on the master's side. For example, for xm4.org,
it takes 113s to receive about 1500 MB data, i.e. around 13.36 MB/s, 
which is much slower than the maximum speed of both memory and disk. 
Thus, the performance are much similar. We also notice that for some
queries such as xm3.org and xm5.org, in-memory case is even a bit 
slower than on-disk case, one possible explanation is that the time 
was taken by calling System.gc(). 


\item \textbf{The execution time is steady}

Compared with the ADBIS study, the execution time of each run is more
steady. My explanation to this result is that for very large scale
of data, the fluctuation has a weaker influence on the execution
time, which increased from milliseconds to seconds. 


\end{itemize}


 
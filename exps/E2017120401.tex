\subsection{E2017120401: Base Performance Evaluation}
 
\subsubsection{Purpose}
 
\subsubsection{Settings} 
Hardware: HaoDesk (see \ref{HaoDesk})\\
Software: BaseX 6.8.7, Java 1.8.0\_151.

\textbf{XML Dataset}

An XMark dataset xmark600.xml sized 66.9 GB. A BaseX databases 'xmark600' is
created by the server using command:

\verb|create db xmark600 xmark600.xml|

\textbf{Queries} 

xm1 -- xm6 used in ADBIS submission attached at the bottom.

\subsubsection{Experiment Design}  

A BaseX instance Server first runs in server mode started by the following
command on HaoDesk.

\verb| java -Xmx4g -xms2g -cp BaseX897.jar org.basex.BaseXServer|

Note the databases in Server are all not in main memory mode.

Then, a java program APP runs on COM in charge of sending an input query to
Server via local network and saving results returned from Server to memory(short
for mem) or disk depending on settings. An input query Query that will be
processed in APP is first rewritten into the following XQuery expression:

\verb|for $node in db:open('xmark600')Query return $node|

The results are stored either in memory or on disk depending on APP's settings:
a) memory means the results are stored in memory and then discarded after the
experiments. b) disk means the results are stored in disk and will be preserved
after the experiments.


\subsubsection{Experiment Results}

\textbf{Timing} The execution time is measured in APP. The time period between
starting sending a query and finishing receiving the results is measured as
execution time. Each query is evaluated 5 times. 

\textbf{Process Results}
We removed the results of the  first run and take the average of the rest as
the final execution time listed in Table~\ref{table:E2017120401_1}. 


\begin{table}[t]
	\caption{Experiment Results.}
	\label{table:E2017120401_1}
	\centering
	\begin{tabular}{c|c|c|r}
 		\hline \hline
 query  & storage & time(s)  &   result size  \\
 \hline \hline
 xm1.org &  disk   & 3257.13  & 60,048,845,586 \\
         & memory  &    N/A   &  \\
 \hline
 xm2.org &  disk   &    0.01  &              0 \\
         & memory  &    0.00  &  \\
  \hline
 xm3.org &  disk   &  105.11  &    922,270,281 \\
         & memory  &  107.21  &  \\
  \hline
 xm4.org &  disk   &  126.00  &  1,583,959,305 \\
         & memory  &  112.32  &  \\
  \hline
 xm5.org &  disk   &  112.94  &    989,346,990 \\
         & memory  &  106.88  &   \\
  \hline
 xm6.org &  disk   &   78.08  &  1,351,708,787 \\
         & memory  &   76.21  &   \\
 \hline \hline
	\end{tabular}
\end{table}


Note: The result of xm2.org is always empty (still under investigation).


\subsubsection{Observations}
\textbf{Storage has small influence on execution time}

I noticed one thing that the execution time is pretty similar for
xm3.org and xm5.org. This is because the bottleneck is on the worker's
side but not on the master's side. For example, for xm3.org,
it takes 105s to receive about 880MB data, i.e. around 8MB/s, which
is much slower than the speed of both memory and disk. Thus,
the performance are much similar.

\textbf{The execution time is steady}

Compared with the ADBIS study, the execution time is much more
steady. My explanation to this result is that due to the large scale
of data, the fluctuation has a weaker influence on the execution
time (milliseconds -> seconds). For you reference, the results and
summary are saved in matsu-lab99:/home2/hao/fragmentation/20171202)
By the way, since the results are still not fully correct, I will
commit my code later after solve them.

 